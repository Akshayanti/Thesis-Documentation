\chapter*{Conclusion}
\addcontentsline{toc}{chapter}{Conclusion}

Although the official title of the research seeks to deal with inconsistencies, this work bounces between error detection and correction, and inconsistency detection and correction. We started with an introduction of an evaluation method to check the POS annotation quality of treebanks in a language, followed by an inspection of an inconsistency detection tool (LISCA) and if it can be extended to be used for low resource data. We also tried to correct the problematic instances identified by researchers before us. While some of the attempts at the solutions have been successful, others still need refinement and additional work to complete them, owing to their time requirements.

As the cost of storage falls lower, the size of the treebanks would increase. Essentially, at one point it might be impossible for human annotators to be part of the error-identification and error-correction process for the entire treebank. The current work is primarily aimed at finding the methods that don't need human annotators in the pipeline, and can be relied upon to fix the errors across different languages in a reliable manner. The research has been in some aspect successful at that front.

One major advantage of an iterative process, with respect to UD treebanks, is how individual error types can be focused on in each iteration. With the UD validator (cf. Level 5 checks in \verb|validate.py|\footnote{\url{https://github.com/UniversalDependencies/tools}} file) identifying and notifying the development teams of the individual errors, the process no longer suffers from a cold start problem. There is a high chance that with upcoming iterations, more and more of the experiments discussed in the document would be rendered obsolete for new treebanks, but they are still necessary to fix the issues in the present treebanks.

It is important to note here that the different problems listed in this thesis document rarely occur in isolation. More often than not, many of the problems are intertwined with each other, resulting in error propagation. Having said that, the corrections are also propagated in a similar fashion, whereby finding and correcting the right error solves multiple intertwined issues at the same time. Consider the example of experiment on \texttt{conj\_head} (in Chapter \ref{chap:conj_head}). Correction of this error instance in the specific case of \verb|eu| also corrected the case of falsely annotated non-projectivities in the trees.

Of the problems mentioned in the chapter titled `Future Work Recommendations' (Chapter \ref{chap:future}), there are some that were not worked on at all in the current research and are left for future researchers (For example, \texttt{nmod4obl} in Section  \ref{sec:probnmod4obl}). Additionally, some other problems are still being worked on, and thus do not fall into the scope of the current thesis document (For example, \texttt{FalseNonProjective} and \texttt{auxHead} in Sections \ref{future:nonproj} and \ref{future:auxHead} respectively).

The author hopes that the future researchers will be able to tackle the problems as listed in this thesis in a greater capacity, and improve upon the methods already discussed in this research wherever possible.