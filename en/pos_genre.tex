\verb|pl|-LFG treebank in UDv2.5 (rated 4 stars on a scale of 5 stars) contains data from 10 different genres\footnote{For understanding of what genre category involves exactly what kind of data, refer to the github page of the treebank at \url{https://github.com/UniversalDependencies/UD_Polish-LFG}}. The sentence counts of different genres are shown in Table \ref{tab:genre_pl}. Of the different kind of data in \textit{spoken} genre, we keep only the \textit{conversational}, discarding the others from consideration. We also remove \textit{academic}, \textit{blog} and \textit{legal} data from our consideration owing to a considerably low number of sentences. Table \ref{tab:genre_fi} shows the genre distribution in UDv2.5 \texttt{fi}-TDT data. The genres we work with are marked in bold in the table. 

\begin{table}[H]
    \centering
    \begin{tabular}{|l|c|l|}
        \hline
        \textbf{Genre} & \textbf{Sentence Count} & \textbf{Avg()}\\
        \hline
        \hline
        \textbf{fiction} & 7 252 & 7.124 \\
        \textbf{news} & 6 744 & 8.401\\
        \textbf{nonfiction} & 1 273 & 7.719\\
        \textbf{social} & 526 & 6.977\\
        \textbf{spoken} & 1253 & 6.047\\
        academic & 51 & 8.118\\
        blog & 136 & 7.772\\
        legal & 11 & 9.273\\ 
        \hline
    \end{tabular}
    \caption{Genre Distribution in UDv2.5 \texttt{pl}-LFG treebank}
    \label{tab:genre_pl}
\end{table}

\begin{table}[H]
    \centering
    \begin{tabular}{|l|c|l|}
        \hline
        \textbf{Genre} & \textbf{Sentence Count} & \textbf{Avg()}\\
        \hline
        \hline
        \textbf{fiction} & 2739 & 11.981\\
        \textbf{wiki} & 2269 & 14.049\\
        \textbf{grammar} & 2002 & 8.48\\
        \textbf{blog} & 1781 & 12.533\\
        \textbf{legal} & 1141 & 20.968\\
        \textbf{news} & 3064 & 13.026\\ 
        europarl & 1082 & 18.441\\
        uni\_articles & 1058 & 13.261\\
        \hline
    \end{tabular}
    \caption{Genre Distribution in UDv2.5 \texttt{fi}-TDT treebank}
    \label{tab:genre_fi}
\end{table}

In order to establish that the different genres are annotated consistently within themselves, we downsample the dataset for each genre in \texttt{fi}-TDT treebank to 900 sentences. On this downsampled data, we perform 2-fold cross validation split, and calculate the $\theta_{pos}$ score for the splits. We repeat this calculation 100 times, such that the data is downsampled differently each time, as per a different seed value. Table \ref{tab:genre_all_kfold} shows the calculated $\theta_{pos}$ scores averaged over 100 different runs.

\begin{table}[H]
    \centering
    \begin{tabular}{|l|l|l|}
        \hline
        \textbf{Genres} & \textbf{$\theta_{pos}$} ($\pm$ sd) & $\Theta_{pos}$\\
        \hline
        \textbf{fiction} & 0.316 $\pm$ 0.015 & 0.5\\
        \textbf{wiki} & 0.3 $\pm$ 0.017 & 0.5\\
        \textbf{grammar} & 0.427 $\pm$ 0.021 & 0.5\\
        \textbf{blog} & 0.332 $\pm$ 0.017 & 0.5\\
        \textbf{legal} & 0.216 $\pm$ 0.035 & 0.5\\
        \textbf{news} & 0.286 $\pm$ 0.015 & 0.5\\
        europarl & 0.233 $\pm$ 0.017 & 0.5\\
        uni\_articles & 0.3 $\pm$ 0.014 & 0.5\\
        \hline
    \end{tabular}
    \caption{$\theta_{pos}$ ($\pm$ sd) Scores Averaged Over 100 Different Runs for Different Genres in UDv2.5 \texttt{fi}-TDT Treebank To Show Intra-Genre Annotation Consistency}
    \label{tab:genre_all_kfold}
\end{table}

As can be seen from Table \ref{tab:genre_all_kfold}, the different genres in the treebank are internally consistent in their annotation, as per the constraint as per Equation \ref{eqn:size_constraint}.

We start the inter-genre analysis by downsampling the datasets for different genres in the dataset. Table \ref{tab:downsample_genre} shows the count of sentences in the downsampled data, along with the F-measure score associated with each genre. Each genre is downsampled to the number of instances such that the condition as specified in Equation \ref{eqn:size_constraint} is satisfied. The F-measure scores reported in the table correspond to the calculated F-measure score for the downsampled data, averaged over 100 different runs with different seed values.

\begin{table}[H]
    \centering
    \scalebox{0.8}{
    \begin{tabular}{|c|l|c|c|}
        \hline
        \textbf{Language} & \textbf{Genre (X)} & \textbf{Downsampled To} & \textbf{$size(X) \cdot \frac{Avg(X)}{Avg(A)}$}\\
         \hline
         & fiction & 500 & 424 \\
         & \textbf{news} & 500 & 500 \\
        \texttt{pl} & nonfiction & 500 & 459 \\
         & social & 500 & 415 \\
         & spoken & 600 & 432 \\
        \hline
         & fiction & 1000 & 571 \\
         & wiki & 1000 & 670 \\
        \texttt{fi} & grammar & 1000 & 404 \\
         & blog & 1000 & 598 \\
         & \textbf{legal} & 1000 & 1000 \\
         & news & 1000 & 621 \\
        \hline
    \end{tabular}
    }
    \caption[Counts of Sentences for Different Genres in Downsampled Data from UDv2.5 \texttt{fi}-TDT and \texttt{pl}-LFG Treebanks]{Counts of Sentences and F-measure Scores ($\pm$ sd) Averaged over 100 Different Runs for Different Genres in Downsampled Data from UDv2.5 \texttt{fi}-TDT and \texttt{pl}-LFG Treebanks. $A$ in $Avg(A)$ in the third column refers to the genre with the highest number of average words per sentence in each language, marked in bold.}
    \label{tab:downsample_genre}
\end{table}

For each of the downsampled data, we compute the $\theta_{pos}$ scores across different genres. We present the scores for \texttt{pl} data in Table \ref{tab:inter_genre-pl} and for \texttt{fi} data in Table \ref{tab:inter_genre-fi}. It is worth noting that for most genres, the $\Theta_{pos}$ constraint as employed in Equation \ref{eqn:size_constraint} isn't enough, as $\theta_{pos}$ frequently surpasses the imposed limit of $0.5$.

\begin{table}[H]
\centering
\scalebox{1.0}{
    \begin{tabular}{|l|l|l|l|l|}
    \hline
    \textbf{Genres} & \textbf{news} & \textbf{nonfiction} & \textbf{social} & \textbf{spoken}\\
    \hline
    \textbf{fiction} & 0.754 $\pm$0.047 & 0.556 $\pm$0.028 & 0.726 $\pm$0.032 & 1.059 $\pm$0.047\\
    \textbf{news} & - & 0.55 $\pm$0.032 & 0.906 $\pm$0.044 & 1.53 $\pm$0.071\\
    \textbf{nonfiction} & - & - & 0.624 $\pm$0.027 & 1.285 $\pm$0.046\\
    \textbf{social} & - & - & - & 1.178 $\pm$0.033\\
    \hline
\end{tabular}}
\caption{$\theta_{pos}$ Scores ($\pm$ sd) Averaged over 100 runs for Inter-Genre Analysis in Downsampled UDv2.5 \texttt{pl}-LFG Data}
\label{tab:inter_genre-pl}
\end{table}

\begin{table}[H]
\centering
\scalebox{0.85}{
    \begin{tabular}{|l|l|l|l|l|l|}
    \hline
    \textbf{Genres} & \textbf{blog} & \textbf{grammar} & \textbf{wiki} & \textbf{legal} & \textbf{news}\\
    \hline
    \textbf{fiction} & 0.356 $\pm$0.014 & 0.47 $\pm$0.019 & 1.552 $\pm$0.041 & 1.559 $\pm$0.04 & 1.323 $\pm$0.044\\
    \textbf{blog} & - & 0.504 $\pm$0.018 & 1.307 $\pm$0.042 & 1.328 $\pm$0.026 & 1.113 $\pm$0.043\\
    \textbf{grammar} & - & - & 1.166 $\pm$0.041 & 1.554 $\pm$0.036 & 0.888 $\pm$0.035\\
    \textbf{wiki} & - & - & - & 1.229 $\pm$0.032 & 0.473 $\pm$0.021\\
    \textbf{legal} & - & - & - & - & 1.078 $\pm$0.026\\
    \hline
\end{tabular}}
\caption{$\theta_{pos}$ Scores ($\pm$ sd) Averaged over 100 runs for Inter-Genre Analysis in Downsampled UDv2.5 \texttt{fi}-TDT Data}
\label{tab:inter_genre-fi}
\end{table}

\newpage

For the experiment, we first split \verb|pl|-LFG treebank into the constituent genres, discarding those that are not marked in bold in Table \ref{tab:genre_pl}. For the data from considered genres, we proceed as follows:

\begin{enumerate}
    \item For a given genre, we randomly sample 125 instances from the genre data. We refer to this resultant split as working data for the genre.
    \item For the given genre, compute $\theta_{pos}$ score of the working data of this genre with the working data of other genres.
    \item Report the computed $\theta_{pos}$ scores.
\end{enumerate}

We repeat the above steps for a 100 times, generating randomly sampled data for each genre, every time. We report the average of the computed scores from all the runs in Table \ref{tab:klcpos3-confusion-results}. For an individual genre under consideration, we 


Intuitively, the pair of genres with the smaller value of $\theta_{pos}$ metric are more similar to each other than those contained in the pair with a higher metric value. From the table, it is evident that the \textit{spoken} genre is least similar to any other considered genre, followed by \textit{social} and \textit{blog}. 


\newpage