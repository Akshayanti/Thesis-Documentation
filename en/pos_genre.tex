\subsection{Inter-Genre Similarity}

\verb|pl|-LFG treebank in UDv2.5 (rated 4 stars on a scale of 5 stars) contains data from 8 different genres\footnote{For understanding of what genre category involves exactly what kind of data, refer to the github page of the treebank at \url{https://github.com/UniversalDependencies/UD_Polish-LFG}}. The sentence counts of different genres are shown in Table \ref{tab:genre_pl}. We club together the different kind of data in \textit{spoken} genre, as one. We remove \textit{academic}, \textit{blog} and \textit{legal} data from our consideration owing to a considerably low number of sentences. Table \ref{tab:genre_fi} shows the genre distribution in UDv2.5 \texttt{fi}-TDT data. In this case, the data with source as \textit{europarl} and university articles (\textit{uni\_articles}) is kept separate from other categories. The genres we work with are marked in bold in the table. 

\begin{table}[H]
    \centering
    \begin{tabular}{|l|c|l|}
        \hline
        \textbf{Genre} & \textbf{Sentence Count} & \textbf{Avg()}\\
        \hline
        \hline
        \textbf{fiction} & 7 252 & 7.124 \\
        \textbf{news} & 6 744 & 8.401\\
        \textbf{nonfiction} & 1 273 & 7.719\\
        \textbf{social} & 526 & 6.977\\
        \textbf{spoken} & 1253 & 6.047\\
        academic & 51 & 8.118\\
        blog & 136 & 7.772\\
        legal & 11 & 9.273\\ 
        \hline
    \end{tabular}
    \caption{Genre Distribution in UDv2.5 \texttt{pl}-LFG treebank}
    \label{tab:genre_pl}
\end{table}

\begin{table}[H]
    \centering
    \begin{tabular}{|l|c|l|}
        \hline
        \textbf{Genre} & \textbf{Sentence Count} & \textbf{Avg()}\\
        \hline
        \hline
        \textbf{fiction} & 2739 & 11.981\\
        \textbf{wiki} & 2269 & 14.049\\
        \textbf{grammar} & 2002 & 8.48\\
        \textbf{blog} & 1781 & 12.533\\
        \textbf{legal} & 1141 & 20.968\\
        \textbf{news} & 3064 & 13.026\\ 
        europarl & 1082 & 18.441\\
        uni\_articles & 1058 & 13.261\\
        \hline
    \end{tabular}
    \caption{Genre Distribution in UDv2.5 \texttt{fi}-TDT treebank}
    \label{tab:genre_fi}
\end{table}

In order to establish that the different genres are annotated consistently within themselves, we downsample the dataset for each genre in \texttt{fi}-TDT treebank to 900 sentences. On this downsampled data, we perform 2-fold cross validation split, and calculate the $\theta_{pos}$ score for the splits. We repeat this calculation 100 times, such that the data is downsampled differently each time, as per a different seed value. Table \ref{tab:genre_all_kfold} shows the calculated $\theta_{pos}$ scores averaged over 100 different runs.

\begin{table}[H]
    \centering
    \begin{tabular}{|l|l|l|}
        \hline
        \textbf{Genres} & \textbf{$\theta_{pos}$} ($\pm$ sd) & $\Theta_{pos}$\\
        \hline
        \textbf{fiction} & 0.316 $\pm$ 0.015 & 0.5\\
        \textbf{wiki} & 0.3 $\pm$ 0.017 & 0.5\\
        \textbf{grammar} & 0.427 $\pm$ 0.021 & 0.5\\
        \textbf{blog} & 0.332 $\pm$ 0.017 & 0.5\\
        \textbf{legal} & 0.216 $\pm$ 0.035 & 0.5\\
        \textbf{news} & 0.286 $\pm$ 0.015 & 0.5\\
        europarl & 0.233 $\pm$ 0.017 & 0.5\\
        uni\_articles & 0.3 $\pm$ 0.014 & 0.5\\
        \hline
    \end{tabular}
    \caption{$\theta_{pos}$ ($\pm$ sd) Scores Averaged Over 100 Different Runs for Different Genres in UDv2.5 \texttt{fi}-TDT Treebank To Show Intra-Genre Annotation Consistency}
    \label{tab:genre_all_kfold}
\end{table}

As can be seen from Table \ref{tab:genre_all_kfold}, the different genres in the treebank are internally consistent in their annotation, as per the constraint in Equation \ref{eqn:size_constraint}.

We start the inter-genre analysis by downsampling the datasets for different genres in the dataset. Table \ref{tab:downsample_genre} shows the count of sentences in the downsampled data from each genre. Each genre is downsampled to the number of instances such that the condition as specified in Equation \ref{eqn:size_constraint} is satisfied.

\begin{table}[H]
    \centering
    \begin{tabular}{|c|l|c|c|}
        \hline
        \textbf{Language} & \textbf{Genre (X)} & \textbf{Downsampled To} & \textbf{$size(X) \cdot \frac{Avg(X)}{Avg(A)}$}\\
         \hline
         & fiction & 500 & 424 \\
         & \textbf{news} & 500 & 500 \\
        \texttt{pl} & nonfiction & 500 & 459 \\
         & social & 500 & 415 \\
         & spoken & 600 & 432 \\
        \hline
         & fiction & 1000 & 571 \\
         & wiki & 1000 & 670 \\
        \texttt{fi} & grammar & 1000 & 404 \\
         & blog & 1000 & 598 \\
         & \textbf{legal} & 1000 & 1000 \\
         & news & 1000 & 621 \\
        \hline
    \end{tabular}
    \caption[Counts of Sentences for Different Genres in Downsampled Data from UDv2.5 \texttt{fi}-TDT and \texttt{pl}-LFG Treebanks]{Counts of Sentences for Different Genres in Downsampled Data from UDv2.5 \texttt{fi}-TDT and \texttt{pl}-LFG Treebanks. $A$ in $Avg(A)$ in the third column refers to the genre with the highest number of average words per sentence in each language, marked in bold.}
    \label{tab:downsample_genre}
\end{table}

For downsampled data from each genre, we compute the $\theta_{pos}$ scores. We present the scores for \texttt{pl} data in Table \ref{tab:inter_genre-pl} and for \texttt{fi} data in Table \ref{tab:inter_genre-fi}. It is worth noting that for most genres, the $\Theta_{pos}$ constraint as employed in Equation \ref{eqn:size_constraint} isn't enough, as $\theta_{pos}$ frequently surpasses the imposed limit of $0.5$.

\begin{table}[H]
\centering
\scalebox{1.0}{
    \begin{tabular}{|l|l|l|l|l|}
    \hline
    \textbf{Genres} & \textbf{news} & \textbf{nonfiction} & \textbf{social} & \textbf{spoken}\\
    \hline
    \textbf{fiction} & 0.754 $\pm$ 0.047 & 0.556 $\pm$ 0.028 & 0.726 $\pm$ 0.032 & 1.059 $\pm$ 0.047\\
    \textbf{news} & - & 0.55 $\pm$ 0.032 & 0.906 $\pm$ 0.044 & 1.53 $\pm$ 0.071\\
    \textbf{nonfiction} & - & - & 0.624 $\pm$ 0.027 & 1.285 $\pm$ 0.046\\
    \textbf{social} & - & - & - & 1.178 $\pm$ 0.033\\
    \hline
\end{tabular}}
\caption{$\theta_{pos}$ Scores ($\pm$ sd) Averaged over 100 runs for Inter-Genre Analysis in Downsampled UDv2.5 \texttt{pl}-LFG Data}
\label{tab:inter_genre-pl}
\end{table}

\begin{table}[H]
\centering
\scalebox{0.85}{
    \begin{tabular}{|l|l|l|l|l|l|}
    \hline
    \textbf{Genres} & \textbf{blog} & \textbf{grammar} & \textbf{wiki} & \textbf{legal} & \textbf{news}\\
    \hline
    \textbf{fiction} & 0.356 $\pm$ 0.014 & 0.47 $\pm$ 0.019 & 1.552 $\pm$ 0.041 & 1.559 $\pm$ 0.04 & 1.323 $\pm$ 0.044\\
    \textbf{blog} & - & 0.504 $\pm$ 0.018 & 1.307 $\pm$ 0.042 & 1.328 $\pm$ 0.026 & 1.113 $\pm$ 0.043\\
    \textbf{grammar} & - & - & 1.166 $\pm$ 0.041 & 1.554 $\pm$ 0.036 & 0.888 $\pm$ 0.035\\
    \textbf{wiki} & - & - & - & 1.229 $\pm$ 0.032 & 0.473 $\pm$ 0.021\\
    \textbf{legal} & - & - & - & - & 1.078 $\pm$ 0.026\\
    \hline
\end{tabular}}
\caption{$\theta_{pos}$ Scores ($\pm$ sd) Averaged over 100 runs for Inter-Genre Analysis in Downsampled UDv2.5 \texttt{fi}-TDT Data}
\label{tab:inter_genre-fi}
\end{table}

We attempted to associate the $\theta_{pos}$ scores across two genres based on if the genre belonged to spoken discourse, or from textual medium. However, as can be seen from the tables above, the scores can not be estimated on the basis of such distinction.

Looking at the data in the tables above, the maximal $\theta_{pos}$ score of $1.559$ is computed between \textit{legal} and \textit{fiction} categories. We hypothesise that a combination of F-score metric with a metric on `Narrative vs Non-Narrative Concerns' can be used to explain such high score. However, there exists no numeric metric to compute a genre's score on its `Narrative vs Non-Narrative Concerns' to the best of our knowledge. We therefore, are unable to associate the upper limit on $\theta_{pos}$ scores with respect to individual genres.

However, we can estimate a general upper bound. We allow some room for change in $\theta_{pos}$ score owing to high quality of annotation as while accounting for variability of dataset size change. With that in mind, we frame the general upper bound on $\theta_{pos}$ scores between genre $x$ in dataset $A$ (written as $A_{x}$) and genre $y$ in dataset $B$ (written as $B_{y}$) as in Equation \ref{eqn:theta_pos_genre_limit}, given below:

\begin{equation}
    \boxed{\theta_{pos}(A_{x}, B_{y}) \leq \Theta_{pos}(A_{x}, B_{y}) = 2.0}
\label{eqn:theta_pos_genre_limit}
\end{equation}
\newpage